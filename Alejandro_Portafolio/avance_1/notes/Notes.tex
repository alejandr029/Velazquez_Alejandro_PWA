
% Intended LaTeX compiler: pdflatex
\documentclass[12pt,a4paper, twosite]{article}
\usepackage[utf8]{inputenc}
\usepackage[T1]{fontenc}
\usepackage{graphicx}
\usepackage{grffile}
\usepackage{longtable}
\usepackage{wrapfig}
\usepackage{rotating}
\usepackage[normalem]{ulem}
\usepackage{amsmath}
\usepackage{textcomp}
\usepackage{amssymb}
\usepackage{capt-of}
\usepackage{hyperref}
\usepackage[left=2.00cm, right=2.50cm, top=2.50cm, bottom=2.00cm]{geometry}
\usepackage{fancyhdr}
\usepackage{verbatim}

\renewcommand{\headrulewidth}{1.0pt}
\pagestyle{fancy}
\date{}
\title{Notes}
\hypersetup{
 pdfauthor={Velazquez Gonzalez Jesus Alejandro},
 pdftitle={Notes},
 pdfkeywords={},
 pdfsubject={},
 pdfcreator={Emacs 26.2 (Org mode 9.1.9)}, 
 pdflang={English}}
\begin{document}

\maketitle
\tableofcontents

\newpage

\section{Notes}
\subsection{Functions}

\begin{verbatim}
  // Calculate the volume of a prism using parameters length, width, and height
  var prism = function(l,w,h)
  {
      return l * w * h;
  }

  // Print the result of the prism function with specific values
  console.log(prism(34,23,2));

  // Define a curried function prisma to calculate the volume of a prism
  var prisma = function (l){
      return function(w){
          return function(h){
              return l * w * h;
          }
      }
  }
  // Print the result of the prisma function with curried values
  console.log(prisma(20)(12)(12));

  // Immediately-invoked function expression (IIFE) to log a message
  var foo = (function(){
      console.log("Here I am");
  }());

  // Constant variable bar storing the result of an IIFE
  const bar = (function() {
      return 56;
  }());

  // Print the value of the constant variable bar
  console.log(bar)
\end{verbatim}

\newpage

\subsection{Named Function}
\begin{verbatim}
  // Define a named function sum and an anonymous function sumaEnoy for addition
  var suma = function sum (a,b){
    return a + b;
  }
  var sumaEnoy = function (a,b){
    return a + b;
  }

  // Print the result of the named function sum and the anonymous function sumaEnoy
  console.log(suma(1,2));
  console.log(sumaEnoy(1,2));

  // Attempt to call the undefined function sum (commented out)
  // console.log(sum(1,2)); //(Is not Undefinated)

  // Attempt to call the undefined function foo
  foo();
  // Define foo as an anonymous function and call it
  var foo = function (){
      console.log("example");
  }

  // Call the function foo successfully
  foo();

  // Declare function foo and call it before its declaration
  function foo (){
      console.log("example");
  }

  // Define a function say and call it recursively with decreasing times argument
  var say = function say(times){
      if(times > 0 ){
          say = undefined;
          console.log("hello");
          say(times - 1 );
      }
  }

  // Assign the say function to saysay, then try to call the undefined say function
  var saysay = say; //say is not a function
  say = "opps";
  saysay(10);
\end{verbatim}

\newpage

\subsection{Variadic Functions}

\begin{verbatim}
  // Define a variadic function persons to log messages for a person
  function persons(per, ...msg){
    msg.forEach(arg => {
        console.log(per + " say: " + arg);
    })
  }

  // Call the persons function with various arguments including an anonymous function
  obj = {
    username : "bar",
    status: true
  }
  persons("fer","hello","world","js","React", ((x)=> x * x)(2))

  // Define and call an anonymous function foo
  var foo = function(){
    console.log("example");    
  }
  foo();

  // Define a function foo with a callArg parameter, call recursively with true
  var foo = function(callArg){
    console.log("messages");

    if(callArg == true) foo(false)
  }
  foo(true)

  // Define a function printMsg with a default parameter msg
  function printMsg(msg ='default messages'){
    console.log(msg);
  }

  // Call printMsg with different arguments
  printMsg("example")
  printMsg(undefined)
  printMsg(null)
  printMsg()
\end{verbatim}

\newpage

\subsection{Fetch Structure Examples}

\begin{verbatim}
  // Fetch data from the PokeAPI and log the JSON response
  fetch('https://pokeapi.co/api/v2/pokemon/ditto')
  .then(resp => resp.json())
  .then(json => console.log(json))
  .catch(err => console.log(err) )

  // Set request headers in the fetch request
  fetch('',{ 
      headers: new Headers({
          'Accept': "text/plain",
          'x-your-custom-header':  'example.values'
      })
  })

  // Fetch data from JSONPlaceholder and log details from the response
  var url ='https://jsonplaceholder.typicode.com/posts';
  fetch(url)
  .then(response => response.json())
  .then(response => {
      response.forEach(element => {
          console.log("ID" + element.id + " -- Title " + element.title );
      });
  });

  // Fetch data from JSONPlaceholder albums and log details
  var url = 'https://jsonplaceholder.typicode.com/albums'
  fetch(url)
  .then(res => res.json())
  .then(res => res.forEach(element => {
      console.log("userId "+ element.userId + "--- id "+ element.id+ "--- title " + element.title)
  }))
  
  // Create an unordered list element in the HTML body
  const questionList = document.createElement('ul');
  document.body.appendChild(questionList);

  // Fetch data and process the response to access specific properties
  const responseData = fetch(url).then(response => response.json());
  responseData.then(({items, has_more, quota_max, quota_remaining}) => {
      for (const {title, score, owner, link, answer_count} of items) {
      }
  });
\end{verbatim}
\newpage

\subsection{Axios}

\begin{verbatim}
  // Require the Axios library
  const axios = require('axios');

  // Define a URL for Axios to fetch data
  const url = "https://jsonplaceholder.typicode.com/users";

  // Fetch data using Axios and log each element in the response
  axios.get(url).then(res =>
      res.data.forEach(element => {
          console.log(element);
      })
  )

  // Perform a POST request with Axios, sending data and logging the response
  axios.post(url,
      {
          username:"foo bar",
          email: "foo.bar.com"
      }
  ).then(res => console.log(res.data));
\end{verbatim}




\end{document}